% !TEX program = xelatex
\documentclass[UTF8]{ctexart}

\RequirePackage{inputenc}
\RequirePackage{fontspec}
\RequirePackage{xeCJK}

\RequirePackage{amsmath}
\RequirePackage{amssymb}
\RequirePackage{mathpazo}
\RequirePackage{pgfplots}
\RequirePackage{tikz}
\RequirePackage{tkz-euclide}

\RequirePackage[hidelinks]{hyperref}

\RequirePackage{multicol}

\usetikzlibrary{calc}
\usetikzlibrary{intersections}
\usetikzlibrary{angles}
\usetkzobj{all}

\RequirePackage{subfigure}

\setmainfont{Times New Roman}

\setCJKmainfont{等线}
\setCJKsansfont{等线}
\setCJKmonofont{等线}

\let\nvec\vec
\def\vec#1{\nvec{\vphantom b\smash{#1}}}

\newcommand*{\dif}{\mathop{}\!\mathrm{d}}

\renewcommand{\Re}{\operatorname{Re}}
\renewcommand{\Im}{\operatorname{Im}}
\newcommand{\arccot}{\operatorname{arccot}}

\newcommand{\rnum}[1]{\uppercase\expandafter{\romannumeral #1\relax}}

\usepackage{geometry}
\geometry
{
    left=1.25in,
    right=1.25in,
    top=1in,
    bottom=1in
}

\title{关于第07期题目的讨论}
\author{数学学习小组}
\date{2020.07.16}

\begin{document}

\maketitle

\newpage

\tableofcontents

\newpage

\setlength{\parindent}{0pt}
\setlength{\columnseprule}{0.4pt}
\setlength{\columnsep}{40pt}

\section{题目07-1}
    本题来源于第07期(2020.07.16)小组讨论题中,原题号为2018年虹口一模20题的第2小问。\\[3mm]
    对于抛物线$\Gamma$:
    \begin{large}
        \begin{equation*}
            y^2=2px
        \end{equation*}
    \end{large}\\
    显然其准线为$l:x=-\dfrac{p}{2}$。\\[3mm]
    设抛物线$\Gamma$上有一点$A$,过$A$作准线$l$的垂线,垂足为$E$。\\[3mm]
    \begin{figure}[h]
        \begin{center}
            \qquad\qquad\qquad
            \begin{tikzpicture}[scale=0.5]
                \tkzDefPoint(0,0){O}

                \tkzDefPoint(-4,0){X1}
                \tkzDefPoint(+13,0){X2}

                \tkzDefPoint(0,-9){Y1}
                \tkzDefPoint(0,+9){Y2}

                \tkzDrawSegments[->](X1,X2)
                \tkzDrawSegments[->](Y1,Y2)

                \tkzLabelPoint[above](X2) {$x$}
                \tkzLabelPoint[right](Y2) {$y$}
                
                \draw[thick,samples=100,domain=0:7] plot({+0.2*\x^2},{+\x});
                \draw[thick,samples=100,domain=0:7] plot({+0.2*\x^2},{-\x});

                \tkzDefPoint(3.2,4){A}
                \tkzDefPoint(2.5,0){F}
                
                \tkzDefPoint(-2.5,+8){l1}
                \tkzDefPoint(-2.5,-8){l2}

                \tkzDefLine[orthogonal=through A](l1,l2)
                \tkzGetPoint{a}

                \tkzInterLL(l1,l2)(A,a)
                \tkzGetPoint{E}

                \tkzDrawSegment(A,E)
                \tkzDrawSegment(A,F)
                \tkzDrawSegment[dashed](l1,l2)

                \tkzMarkAngle[size=0.6](E,A,F)
                \tkzMarkRightAngle[size=0.35](A,E,l2)

                \tkzDrawPoints[fill=white](O,A,E,F)
                \tkzLabelPoint[above](A){$A$}
                \tkzLabelPoint[below](F){$F$}
                \tkzLabelPoint[left](E){$E$}
                \tkzLabelPoint[below](l2){$l$}
                \tkzLabelSegment[above,pos=0.4](A,E){$d$}
                
            \end{tikzpicture}
            \caption{题目07-01的示意图}
        \end{center}
    \end{figure}\\
    现设$AE=d$,且满足$\dfrac{3p}{4}\leq d\leq\dfrac{4p}{3}$,求解$\angle EAF$的取值范围。

\newpage

\subsection{第一种解法}
    \begin{center}
        整理者:李宇轩
    \end{center}
    \begin{multicols}{2}
        \small
        \textbf{核心思路:利用向量夹角公式。}\\[5mm]
        设$A(x,y)$,已知$F\left(\dfrac{p}{2},0\right)$,且$E\left(-\dfrac{p}{2},y\right)$。\\[5mm]
        $\overrightarrow{EA}=\left(x+\dfrac{p}{2},0\right)$\\[5mm]
        $\overrightarrow{FA}=\left(x-\dfrac{p}{2},y\right)$\\[5mm]
        根据抛物线性质可知:$|\overrightarrow{EA}|=|\overrightarrow{FA}|$\\[5mm]
        $\cos{\angle EAF}=\dfrac{\overrightarrow{EA}+\overrightarrow{FA}}{|\overrightarrow{EA}|\cdot|\overrightarrow{FA}|}$\\[5mm]
        $\cos{\angle EAF}=\dfrac{\overrightarrow{EA}+\overrightarrow{FA}}{|\overrightarrow{EA}|^2}$\\[7mm]
        $\cos{\angle EAF}=\dfrac{\left(x+\dfrac{p}{2}\right)\cdot\left(x-\dfrac{p}{2}\right)}{\left(x+\dfrac{p}{2}\right)^2}$\\[5mm]
        $\cos{\angle EAF}=\dfrac{x-\dfrac{p}{2}}{x+\dfrac{p}{2}}$\\[5mm]
        $\cos{\angle EAF}=1-\dfrac{p}{x+\dfrac{p}{2}}$\\[8mm]
        当$d=\dfrac{3}{4}p$,有$x=\dfrac{1}{4}p$,此时$\cos{\angle EAF}=-\dfrac{1}{3}p$。\\[5mm]
        当$d=\dfrac{4}{3}p$,有$x=\dfrac{5}{6}p$,此时$\cos{\angle EAF}=\dfrac{1}{4}p$。\\[5mm]
        故$\angle EAF\in\left[\arccos{\left(\dfrac{1}{4}\right)},\arccos{\left(-\dfrac{1}{3}\right)}\right]$
        \newpage
    \end{multicols}

\newpage

\subsection{第二种解法}
    \begin{center}
        整理者:李宇轩
    \end{center}
    \begin{multicols}{2}
        \small
        \textbf{核心思路:利用余弦定理。}\\[5mm]
        设$A(x,y)$,已知$F\left(\dfrac{p}{2},0\right)$,且$E\left(-\dfrac{p}{2},y\right)$。\\[5mm]
        在$\triangle EAF$中,$AE=AF=x+\dfrac{p}{2}$,$EF^2=y^2+p^2$\\[5mm]
        $\cos{\angle EAF}=\dfrac{AE^2+AF^2-EF^2}{2\cdot AE\cdot AF}$\\[5mm]
        $\cos{\angle EAF}=\dfrac{2\cdot\left(x+\dfrac{p}{2}\right)^2-\left(y^2+p^2\right)}{2\cdot\left(x+\dfrac{p}{2}\right)^2}$\\[5mm]
        $\cos{\angle EAF}=1-\dfrac{y^2+p^2}{2\cdot\left(x+\dfrac{p}{2}\right)^2}$\\[5mm]
        $\cos{\angle EAF}=1-\dfrac{2px+p^2}{2\cdot\left(x+\dfrac{p}{2}\right)^2}$\\[5mm]
        $\cos{\angle EAF}=1-\dfrac{2p\left(x+\dfrac{p}{2}\right)}{2\cdot\left(x+\dfrac{p}{2}\right)^2}$\\[5mm]
        $\cos{\angle EAF}=1-\dfrac{p}{x+\dfrac{p}{2}}$\\[5mm]
        当$d=\dfrac{3}{4}p$,有$x=\dfrac{1}{4}p$,此时$\cos{\angle EAF}=-\dfrac{1}{3}p$。\\[5mm]
        当$d=\dfrac{4}{3}p$,有$x=\dfrac{5}{6}p$,此时$\cos{\angle EAF}=\dfrac{1}{4}p$。\\[5mm]
        故$\angle EAF\in\left[\arccos{\left(\dfrac{1}{4}\right)},\arccos{\left(-\dfrac{1}{3}\right)}\right]$
        \newpage
    \end{multicols}

\newpage

\subsection{第三种解法}
    \begin{center}
        整理者:白昱轩
    \end{center}
    \begin{multicols}{1}
        \small
        \textbf{核心思路:说理论证。}\\[5mm]
        因为$EA\parallel x$,故$\angle EAF$与$\angle AFO$互补。\\[5mm]
        因为点$F$移动的过程中:\\[5mm]
        显然$\angle AFO$随$x$的变大而变大。\\[5mm]
        所以$\angle EAF$随$x$的变大而变小。\\[5mm]
        故$d$最小时$\angle EAF$最大。\\[5mm]
        而$d$最大时$\angle EAF$最小。\\[5mm]
        当d=$\dfrac{3p}{4}$时,$\cos \angle EAF = \left(\dfrac{\dfrac{3p}{4}-p}{\dfrac{3p}{4}}  \right) = -\dfrac{1}{3}$\\[3mm]
        当d=$\dfrac{4p}{3}$时,$\cos \angle EAF = \left(\dfrac{\dfrac{4p}{3}-p}{\dfrac{4p}{3}}  \right) = \dfrac{1}{4}$\\[3mm]
        故$\angle EAF \in \left[\arccos \dfrac{1}{4} , \arccos \dfrac{-1}{3}\right]$\\[10mm]
        \newpage
    \end{multicols}

\newpage

\section{题目07-2}
    本题来源于第07期(2020.07.16)小组讨论题中,原题号为2018年虹口一模20题的第3小问。\\[3mm]
    对于抛物线$\Gamma$:
    \begin{large}
        \begin{equation*}
            y^2=2px
        \end{equation*}
    \end{large}\\
    显然其准线为$l:x=-\dfrac{p}{2}$。\\[3mm]
    设抛物线$\Gamma$上有一点$A$,过$A$作准线$l$的垂线,垂足为$E$。\\[3mm]
    \begin{figure}[h]
        \begin{center}
            \qquad\qquad\qquad
            \begin{tikzpicture}[scale=0.5]
                \tkzDefPoint(0,0){O}

                \tkzDefPoint(-4,0){X1}
                \tkzDefPoint(+13,0){X2}

                \tkzDefPoint(0,-9){Y1}
                \tkzDefPoint(0,+9){Y2}

                \tkzDrawSegments[->](X1,X2)
                \tkzDrawSegments[->](Y1,Y2)

                \tkzLabelPoint[above](X2) {$x$}
                \tkzLabelPoint[right](Y2) {$y$}
                
                \draw[thick,samples=100,domain=0:7] plot({+0.2*\x^2},{+\x});
                \draw[thick,samples=100,domain=0:7] plot({+0.2*\x^2},{-\x});

                \tkzDefPoint(3.2,4){A}
                \tkzDefPoint(2.5,0){F}
                
                \tkzDefPoint(-2.5,+8){l1}
                \tkzDefPoint(-2.5,-8){l2}

                \tkzDefLine[orthogonal=through A](l1,l2)
                \tkzGetPoint{a}

                \tkzInterLL(l1,l2)(A,a)
                \tkzGetPoint{E}

                \tkzDrawSegment(A,E)
                \tkzDrawSegment(A,F)
                \tkzDrawSegment[dashed](l1,l2)

                \tkzMarkAngle[size=0.6](E,A,F)
                \tkzMarkRightAngle[size=0.35](A,E,l2)

                \tkzDrawPoints[fill=white](O,A,E,F)
                \tkzLabelPoint[above](A){$A$}
                \tkzLabelPoint[below](F){$F$}
                \tkzLabelPoint[left](E){$E$}
                \tkzLabelPoint[below](l2){$l$}
                \tkzLabelSegment[above,pos=0.4](A,E){$d$}
                
            \end{tikzpicture}
            \caption{题目07-02的示意图}
        \end{center}
    \end{figure}\\
    现判断$\angle EAF$的角平分线所在直线与曲线的交点个数,并说明理由。

\newpage

\subsection{第一种解法}
    \begin{center}
        整理者:白昱轩
    \end{center}
    {
        \small
        \textbf{核心思路:说理论证。}\\[5mm]
        因为$EA=FA$,故$\angle EAF$的角平分线即$EF$的垂直平分线。\\[5mm]
        角平分线已经与抛物线交与点$A$,假设存在第二个交点$A_1$,假设其在$l$上的投影为$E_1$\\[5mm]
        根据抛物线性质:$A_1F=A_1E_1$。\\[5mm]
        根据垂直平分线:$A_1F=A_1E$。\\[5mm]
        由于$\triangle A_1E_1E$始终为直角三角形,且$A_1E_1$为斜边,而$A_1E$为直角边。\\[5mm]
        故根据直角三角形的性质,必然有$A_1E_1<A_1E$,然而与上述结论产生矛盾。\\[5mm]
        故假设不成立,不存在第二个交点,因此只有一个交点。\vspace{2pt}
    }
    \begin{figure}[h]
        \begin{center}
            \qquad\qquad\qquad
            \begin{tikzpicture}[scale=0.4]
                \tkzDefPoint(0,0){O}

                \tkzDefPoint(-4,0){X1}
                \tkzDefPoint(+13,0){X2}

                \tkzDefPoint(0,-9){Y1}
                \tkzDefPoint(0,+9){Y2}

                \tkzDrawSegments[->](X1,X2)
                \tkzDrawSegments[->](Y1,Y2)

                \tkzLabelPoint[above](X2) {$x$}
                \tkzLabelPoint[right](Y2) {$y$}
                
                \draw[thick,samples=100,domain=0:7] plot({+0.2*\x^2},{+\x});
                \draw[thick,samples=100,domain=0:7] plot({+0.2*\x^2},{-\x});

                \tkzDefPoint(3.2,4){A}
                \tkzDefPoint(2.5,0){F}
                
                \tkzDefPoint(-2.5,+8){l1}
                \tkzDefPoint(-2.5,-8){l2}

                \tkzDefLine[orthogonal=through A](l1,l2)
                \tkzGetPoint{a}

                \tkzInterLL(l1,l2)(A,a)
                \tkzGetPoint{E}

                \tkzDefPoint(0.5079,1.5936){A1}
                \tkzDefPoint(-2.5,1.5936){E1}

                \tkzDrawSegment(A,E)
                \tkzDrawSegment(A1,E1)
                \tkzDrawSegment(A,F)
                \tkzDrawSegment[dashed](l1,l2)
                \tkzDrawSegment(E,F)

                \tkzMarkAngle[size=0.6](E,A,F)
                \tkzMarkRightAngle[size=0.35](A,E,l2)
                \tkzMarkRightAngle[size=0.35](A1,E1,l1)

                \tkzDrawPoints[fill=white](O,A,E,F,A1,E1)
                \tkzLabelPoint[above](A){$A$}
                \tkzLabelPoint[right=2pt](A1){$A_1$}
                \tkzLabelPoint[below](F){$F$}
                \tkzLabelPoint[left=3pt](E){$E$}
                \tkzLabelPoint[left](E1){$E_1$}
                \tkzLabelPoint[below](l2){$l$}
                \tkzLabelSegment[above,pos=0.4](A,E){$d$}
            \end{tikzpicture}
            \caption{第一种解法的配图}
        \end{center}
    \end{figure}\\
    以上是本解法的配图。

\newpage

\subsection{第二种解法}
    \begin{center}
        整理者:白昱轩
    \end{center}
    \begin{multicols}{2}
        \small
        \textbf{核心思路:设法向量联立。}\\[5mm]
        设$A(x_1,y_1)$,且$E\left(-\dfrac{p}{2},y_1\right)$,已知$F\left(\dfrac{p}{2},0\right)$。\\[5mm]
        设角平分线为$l$,设角平分线的法向量为$\vec{n}$。\\[5mm]
        $\vec{n} = \overrightarrow{EF} = \left(p,-y_1\right)$\\[5mm]
        $l:p\left(x-x_1\right)-y_1\cdot\left(y-y_1\right) = 0$\\[5mm]
        联立直线和抛物线:\\[5mm]
        \begin{math}
            \begin{cases}
                ~p\left(x-x_1\right)-y_1\cdot\left(y-y_1\right) = 0\\[1mm]
                ~y^2 = 2px\\[1mm]
                ~y_1^2 = 2px_1\\[1mm]
            \end{cases}
        \end{math}\\[5mm]
        $px-px_1-yy_1+y_1^2 = 0$\\[5mm]
        $\dfrac{1}{2}y^2-\dfrac{1}{2}y_1^2-yy_1+y_1^2 = 0$\\[5mm]
        $\dfrac{1}{2}y^2-yy_1+\dfrac{1}{2}y_1^2=0$\\[5mm]
        $\Delta=0$\\[5mm]
        故只有一个交点。
        \newpage
    \end{multicols}   

\newpage

\subsection{第三种解法}
    \begin{center}
        整理者:白昱轩
    \end{center}
    \begin{multicols}{2}
        \small
        \textbf{核心思路:设斜率求解。}\\[5mm]
        设$A(x_1,y_1)$,且$E\left(-\dfrac{p}{2},y_1\right)$,已知$F\left(\dfrac{p}{2},0\right)$。\\[5mm]
        设角平分线为$l$,设其与$x$轴的交点为$P$。\\[5mm]
        设$\theta=\angle EFP$\\[5mm]
        设$\alpha=\angle APF$\\[5mm]
        显然角平分线$AP$也是直线$EF$的垂直平分线。\\[5mm]
        因此$\angle EFP + \angle APF=\dfrac{\pi}{2}$,即$\theta+\alpha=\dfrac{\pi}{2}$。\\[5mm]
        $k=\tan\alpha$\\[5mm]
        $k=\cot\theta$\\[5mm]
        $k=\dfrac{p}{y_1}$\\[5mm]
        $l:y-y_1=\dfrac{p}{y_1}\left(x-x_1\right)$\\[5mm]
        联立直线和抛物线:\\[5mm]
        \begin{math}
            \begin{cases}
                ~y-y_1 = \dfrac{p}{y_1}\left(x-x_1\right)\\[1mm]
                ~y^2 = 2px\\[1mm]
                ~y_1^2 = 2px_1\\[1mm]
            \end{cases}        
        \end{math}\\[5mm]
        $y-y_1=\dfrac{p}{y_1}x-\dfrac{p}{y_1}x_1$\\[5mm]
        $yy_1-y_1^2=px-px_1$\\[5mm]
        $yy_1-y_1^2=\dfrac{y^2}{2}-\dfrac{y_1^2}{2}$\\[5mm]
        $\dfrac{1}{2}y^2-yy_1 + \dfrac{1}{2}y_1^2 = 0$\\[5mm]
        $\Delta = 0$\\[5mm]
        故只有一个交点。
        \newpage
    \end{multicols}   

\newpage

\section{题目07-3}
    本题来源于第07期(2020.07.16)小组讨论题中,原题号为2018年杨浦一模20题的第2小问。\\[3mm]
    对于抛物线$\Gamma$:
    \begin{large}
        \begin{equation*}
            y^2=4x
        \end{equation*}
    \end{large}\\
    设直线$l$与抛物线$\Gamma$相交于点$A$和点$B$。\\[3mm]
    设定圆$C:(x-5)^2+y^2=16$与直线$l$相切,且切点为$M$。
    \begin{figure}[h]
        \begin{center}
            \begin{tikzpicture}[scale=0.5]
                \tkzDefPoint(0,0){O}

                \tkzDefPoint(-4,0){X1}
                \tkzDefPoint(+13,0){X2}

                \tkzDefPoint(0,-9){Y1}
                \tkzDefPoint(0,+9){Y2}

                \tkzDrawSegments[->](X1,X2)
                \tkzDrawSegments[->](Y1,Y2)

                \tkzLabelPoint[above](X2) {$x$}
                \tkzLabelPoint[right](Y2) {$y$}
                
                \draw[thick,samples=100,domain=0:7] plot({+0.2*\x^2},{+\x});
                \draw[thick,samples=100,domain=0:7] plot({+0.2*\x^2},{-\x});

                \tkzDefPoint(0.8,+2){A}
                \tkzDefPoint(0.8,-2){B}

                \tkzDefPoint(0.8,0){M}
                \tkzDefPoint(3.8,0){C}

                \tkzDrawSegment[add=0.8 and 0.8](A,B)
                \tkzDrawCircle(C,M)

                \tkzDrawPoints[fill=white](A,B,M,C)
            \end{tikzpicture}
            \qquad\qquad\qquad\qquad
            \caption{题目07-03的示意图}
        \end{center}
    \end{figure}\\
    若切点$M$为点$A$和点$B$的中点,求直线$l$的方程。

\newpage

\subsection{第一种解法}
    \begin{center}
        整理者:李宇轩
    \end{center}
    \begin{multicols}{2}
        \small
        \textbf{核心思路:利用相切,利用切点。}\\[5mm]
        设直线$x=ty+b$\\[5mm]
        当$t$不存在时,舍。\\[5mm]
        当$t$存在时:\\[5mm]
        \begin{math}
            \begin{cases}
                ~x=ty+b\\[1mm]
                ~y^2=4x\\[1mm]
            \end{cases}
        \end{math}\\[5mm]
        $y^2=4ty+4b$\\[5mm]
        $y^2-4ty-4b=0$\\[5mm]
        $\Delta=16t^2+16b>0$\\[5mm]
        $y_1\cdot y_2=-4t$\\[5mm]
        $y_1+y_2=4t$\\[5mm]
        $x_1+x_2=4t^2+2b$\\[5mm]
        因为$M\left(\dfrac{x_1+x_2}{2}~,~\dfrac{y_1+y_2}{2}\right)$\\[5mm]
        所以$M\left(2t^2+b,2t\right)$\\[8mm]
        点$C(5,0)$至直线$l$的距离应为$4$保证相切:\\[5mm]
        $l:x=ty+b$\\[5mm]
        $l:x-ty-b=0$\\[5mm]
        $d=\dfrac{|5-b|}{\sqrt{1+t^2}}=4$\\[5mm]
        $16t^2+16=b^2-10b+25$\\[5mm]
        $16t^2=b^2-10b+9$\\[50mm]
        向量$\overrightarrow{CM}$应当于向量$\overrightarrow{AB}$垂直保证切点为$M$:\\[5mm]
        $\overrightarrow{CM}=\left(2t^2+b-5,2t\right)$\\[5mm]
        $\overrightarrow{AB}=\vec{d}=\big(t,1\big)$\\[5mm]
        $\overrightarrow{AB}\cdot\overrightarrow{CM}$\\[5mm]
        $t^2\cdot\left(2t^2+b-5\right)+2t=0$\\[5mm]
        1.若$t=0$:$l:x=1$~或~$l:x=9$\\[5mm]
        2.若$t\neq 0$\\[5mm]
        $2t^2+b-5+2=0$\\[5mm]
        $2t^2+b-=0$\\[5mm]
        $2t^2=-b+3$\\[5mm]
        $16t^2=-8b+24$\\[8mm]
        联立两组结论可得:\\[5mm]
        \begin{math}
            \begin{cases}
                ~16t^2=b^2-10b+9\\[1mm]
                ~16t^2=-8b+24\\[1mm]
            \end{cases}
        \end{math}\\[5mm]
        $b^2-10b+9=-8b+24$\\[5mm]
        $b^2-2b-15=0$\\[5mm]
        \begin{math}
            \begin{cases}
                ~b=5\\[1mm]
                ~t^2=-1\\[1mm]
            \end{cases}~~~~
            \begin{cases}
                ~b=-3\\[1mm]
                ~t^2=3\\[1mm]
            \end{cases}
        \end{math}\\[5mm]
        第一组解显然不合理,第二组解不满足$\Delta$,均舍。
        \newpage
    \end{multicols}   

\newpage

\section{题目07-4}
    本题来源于第07期(2020.07.16)小组讨论题中,原题号为2018年普陀一模20题的第3小问。\\[3mm]
    对于椭圆$\Gamma$:
    \begin{large}
        \begin{equation*}
            \dfrac{x^2}{8}+\dfrac{y^2}{4}=1
        \end{equation*}
    \end{large}\\
    设椭圆$\Gamma$的左焦点为$F_1$,设椭圆$\Gamma$的右焦点为$F_2$。\\[3mm]
    设点$M$点$N$为椭圆上位于$x$轴上方的两点,且满足$\overrightarrow{F_1M}\parallel\overrightarrow{F_2N}$。
    \begin{figure}[h]
        \begin{center}
            \begin{tikzpicture}[scale=0.8]
                \tkzDefPoint(0,0){O}

                \tkzDefPoint(-8,0){X1}
                \tkzDefPoint(+8,0){X2}

                \tkzDefPoint(0,-6){Y1}
                \tkzDefPoint(0,+6){Y2}

                \tkzDrawSegments[->](X1,X2)
                \tkzDrawSegments[->](Y1,Y2)

                \tkzLabelPoint[above](X2) {$x$}
                \tkzLabelPoint[right](Y2) {$y$}
                
                \draw[thick](0,0) ellipse(4.24264 and 3);

                \tkzDefPoint(-3,0){F1}
                \tkzDefPoint(+3,0){F2}

                \tkzDefPoint(-4.1623,+0.5811){M}
                \tkzDefPoint(-2.1622,+2.5811){N}

                \tkzDefPoint(+4.1622,-0.5811){H}

                \tkzDrawSegment(F1,M)
                \tkzDrawSegment(H,N)

                \tkzDrawPoints[fill=white](F1,F2,M,N,H)

                \tkzLabelPoints[above](N)
                \tkzLabelPoints[left](M)
                \tkzLabelPoints[below right](H)

                \tkzLabelPoint[below](F1){$F_1$}
                \tkzLabelPoint[below](F2){$F_2$}
            \end{tikzpicture}
            \caption{题目07-04的示意图}
        \end{center}
    \end{figure}\\
    当$|\overrightarrow{F_2N}|-|\overrightarrow{F_1M}|=\sqrt{6}$时,求直线$F_2N$的方程。

\newpage

\subsection{第一种解法}
    \begin{center}
        整理者:杨骐荣
    \end{center}
    \begin{multicols}{2}
        \small
        \textbf{核心思路:利用相切,利用切点。}\\[5mm]
        延长$NF_2$交椭圆于$H$。\\[5mm]
        根据图像的对称性可知 $\left| {\overrightarrow{F_1M}}  \right | = \left | {\overrightarrow{F_2H}}  \right|$\\[5mm]
        设$N(x_1,y_1)$,设$H(x_2,y_2)$\\[5mm]
        故$l_{NH}:y=k(x-2)$\\[5mm]
        $\left | {\overrightarrow{F_2N}}  \right | =\sqrt[]{(x_1-2)^2+y_1^2} =\sqrt[]{k^2+1} \left | {x_1-2}  \right |$\\[5mm]
        $\left | {\overrightarrow{F_2H}}  \right | =\sqrt[]{(x_2-2)^2+y_2^2} =\sqrt[]{k^2+1} \left | {x_2-2}  \right |$\\[5mm]
        因为$\left | {\overrightarrow{F_2N}}  \right | -\left | {\overrightarrow{F_2H}}  \right |=\sqrt{6} $\\[5mm]
        $\sqrt{k^2+1}\cdot\left | {x_1-2}  \right |-\sqrt{k^2+1}\cdot\left | {x_2-2}  \right |=\sqrt{6} $\\[5mm]
        $\sqrt{k^2+1}\cdot(\left | {x_1-2}  \right |-\left | {x_2-2}  \right |)=\sqrt{6} $\\[5mm]
        由图可知:$x_1<2<x_2$\\[5mm]
        $\sqrt{k^2+1}\cdot(2-x_1-x_2+2)=\sqrt{6}$\\[5mm]
        $\sqrt{k^2+1}\cdot[4-(x_1+x_2)]=\sqrt{6} $\\[100mm]
        \begin{math}
            \begin{cases}
                y=k(x-2) \\[1mm]  
                x^2+2y^2=8 
            \end{cases}
        \end{math}\\[5mm]
        $(2k^2+1)x^2-8k^2x+8k^2-8=0$\\[5mm]
        $\Delta>0$恒成立\\[5mm]
        $x_1+x_2=\dfrac{8k^2}{8k^2+1}$\\[5mm]
        代入可得:$\sqrt{k^2+1}\cdot\left[4-\dfrac{8k^2}{2k^2+1}\right]=\sqrt{6}$\\[5mm]
        求解得到:$k^2=\dfrac{1}{2}$~~或~~$k^2=\dfrac{-5}{6}$ (舍)\\[5mm]
        由图可知$k<0$\\[5mm]
        $k=\dfrac{-\sqrt{2} }{2}$\\[5mm]
        $y=\dfrac{-\sqrt{2} }{2}x+\sqrt{2}$\\[5mm]
        \newpage
    \end{multicols}

\newpage

\section{题目07-5}
    本题来源于第07期(2020.07.16)小组讨论题中,原题号为2018年徐汇一模20题的第3小问。\\[3mm]
    对于椭圆$\Gamma$:
    \begin{large}
        \begin{equation*}
            \dfrac{x^2}{8}+\dfrac{y^2}{1}=1
        \end{equation*}
    \end{large}\\
    设椭圆$\Gamma$的右焦点为$F_2$,过焦点$F_2$作两条相互垂直的直线。\\[3mm]
    第一条直线交椭圆与点$A$和点$B$,其中点为$M$。\\[3mm]
    第二条直线交椭圆与点$C$和点$D$,其中点为$N$。\\[3mm]
    已知直线$MN$恒过定点$R\left(\dfrac{2}{3},0\right)$(此为第2小问的结论)。\\[3mm]
    \begin{figure}[h]
        \begin{center}
            \begin{tikzpicture}[scale=0.8]
                \tkzDefPoint(0,0){O}

                \tkzDefPoint(-8,0){X1}
                \tkzDefPoint(+8,0){X2}

                \tkzDefPoint(0,-6){Y1}
                \tkzDefPoint(0,+6){Y2}

                \tkzDrawSegments[->](X1,X2)
                \tkzDrawSegments[->](Y1,Y2)

                \tkzLabelPoint[above](X2) {$x$}
                \tkzLabelPoint[right](Y2) {$y$}
                
                \draw[thick](0,0) ellipse(4.24264 and 3);

                \tkzDefPoint(3,0){F}
                
                \tkzDefPoint(-2.1623,-2.5811){A}
                \tkzDefPoint(+4.1623,+0.5811){B}
                \tkzDefPoint(+1.6126,+2.7749){C}
                \tkzDefPoint(+3.7208,-1.4415){D}

                \tkzDefMidPoint(A,B)
                \tkzGetPoint{M}

                \tkzDefMidPoint(C,D)
                \tkzGetPoint{N}

                \tkzDrawSegment(A,B)
                \tkzDrawSegment(C,D)
                \tkzDrawSegment(M,N)

                \tkzDrawPoints[fill=white](F,A,B,C,D,M,N)

                \tkzLabelPoint[below left](A){$A$}
                \tkzLabelPoint[right](B){$B$}
                \tkzLabelPoint[below](M){$M$}
                \tkzLabelPoint[above](C){$C$}
                \tkzLabelPoint[right](D){$D$}
                \tkzLabelPoint[right](N){$N$}
            \end{tikzpicture}
            \caption{题目07-05的示意图}
        \end{center}
    \end{figure}\\
    求$\triangle MNF_2$的最大值。

\newpage

\subsection{第一种解法}
    \begin{center}
        整理者:李周
    \end{center}
    \begin{multicols}{2}
        \small
        \textbf{核心思路:拆分三角形。}\\[5mm]
        设$M(x_{1},y_{1})$,设$N(x_{2},y_{2})$。\\[5mm]  
        设$l_{AB}:x=ty+1$\\[5mm]
        设$l_{CD}:x=-\dfrac{1}{t}y+1$\\[5mm]
        根据对称性可知,只需讨论$m>0$的情况。\\[5mm]
        联立椭圆和直线:\\[5mm]
        \begin{math}
            \begin{cases}
                x=ty+1\\[1mm]  
                x^2+2y^2-2=0\\[1mm]
            \end{cases}    
        \end{math}\\[5mm]
        $(t^2+2)y^2+2ty-1=0$\\[5mm]
        因为$F_{2}$在椭圆内,所以必有两个交点,故$\Delta >0$\\[5mm]
        $y_{1}+y_{2}=-\dfrac{2t}{t^2+2}$\\[5mm]
        $x_{1}+x_{2}=\dfrac{4}{t^2+2}$\\[5mm]
        $M(\dfrac{2}{t^2+2},-\dfrac{t}{t^2+2})$\\[5mm]
        $N(\dfrac{2t^2}{1+2t^2},\dfrac{t}{1+2t^2})$(同理可得)\\[8mm]
        $S_{\triangle MNF_{2}}=\dfrac{1}{2}\cdot\left(1-\dfrac{2}{3}\right)\cdot\left|y_{m}-y_{n}\right|$\\[4mm]
        $S_{\triangle MNF_{2}}=\dfrac{1}{6}\left|\dfrac{t}{1+2t^2}+\dfrac{t}{t^2+2}\right|$\\[4mm]
        $S_{\triangle MNF_{2}}=\dfrac{1}{6}\cdot\dfrac{3t^3+3t}{2t^4+5t^2+2}$\\[4mm]
        $S_{\triangle MNF_{2}}=\dfrac{t^3+t}{4t^4+10t^2+4}$\\[4mm]
        $S_{\triangle MNF_{2}}=\dfrac{t+\dfrac{1}{t}}{10+4t^2+\dfrac{4}{t^2}}$\\[20mm]
        令$m=t+\dfrac{1}{t}$,故$m\leq -2$或$m\geq 2$。\\[5mm]
        $S_{\triangle MNF_{2}}=\dfrac{m}{4m^2+2}$\\[5mm]
        $S_{\triangle MNF_{2}}=\dfrac{1}{4m+\dfrac{2}{m}}$\\[2mm]
        因为$m\geq 2$,故$S_{\triangle MNF_{2}}$面积的最大值为$\dfrac{1}{9}$。
        \newpage
    \end{multicols}

\newpage

\subsection{第二种解法}
    \begin{center}
        整理者:李周
    \end{center}
    \begin{multicols}{2}
        \small
        \textbf{核心思路:求解直角三角形的两条边。}\\[5mm]
        设$M(x_{1},y_{1})$,设$N(x_{2},y_{2})$。\\[5mm]  
        设$l_{AB}:x=ty+1$\\[5mm]
        设$l_{CD}:x=-\dfrac{1}{t}y+1$\\[5mm]
        根据对称性可知,只需讨论$m>0$的情况。\\[5mm]
        联立椭圆和直线:\\[5mm]
        \begin{math}
            \begin{cases}
                x=ty+1\\[1mm]  
                x^2+2y^2-2=0\\[1mm]
            \end{cases}    
        \end{math}\\[5mm]
        $(t^2+2)y^2+2ty-1=0$\\[5mm]
        因为$F_{2}$在椭圆内,所以必有两个交点,故$\Delta >0$\\[5mm]
        $y_{1}+y_{2}=-\dfrac{2t}{t^2+2}$\\[5mm]
        $x_{1}+x_{2}=\dfrac{4}{t^2+2}$\\[5mm]
        $M(\dfrac{2}{t^2+2},-\dfrac{t}{t^2+2})$\\[5mm]
        $N(\dfrac{2t^2}{1+2t^2},\dfrac{t}{1+2t^2})$(同理可得)\\[8mm]
        $\left|MF_{2}\right|=\sqrt{(\dfrac{2}{t^2+2}-1)^2+(\dfrac{t}{t^2+2})^2}$\\[5mm]
        $\left|MF_{2}\right|=\dfrac{\sqrt{t^4+t^2} }{t^2+2}$\\[5mm]
        $\left|NF_{2}\right|=\sqrt{(\dfrac{2t^2}{1+2t^2}-1)^2+(\dfrac{t}{1+2t^2})^2}$\\[5mm]
        $\left|NF_{2}\right|=\dfrac{\sqrt{t^2+1} }{2t^2+1}$\\[50mm]
        $S_{\triangle MNF_{2}}=\dfrac{1}{2}\cdot\dfrac{\sqrt{t^6+2t^4+t^2}}{2t^4+5t^2+2}$\\[5mm]
        $S_{\triangle MNF_{2}}=\dfrac{1}{2}\cdot\dfrac{t^3+t}{2t^4+5t^2+2}$\\[5mm]
        $S_{\triangle MNF_{2}}=\dfrac{t^3+t}{4t^4+10t^2+4}$\\[5mm]
        $S_{\triangle MNF_{2}}=\dfrac{t+\dfrac{1}{t}}{10+4t^2+\dfrac{4}{t^2}}$\\[8mm]
        令$m=t+\dfrac{1}{t}$,故$m\leq -2$或$m\geq 2$。\\[5mm]
        $S_{\triangle MNF_{2}}=\dfrac{m}{4m^2+2}$\\[5mm]
        $S_{\triangle MNF_{2}}=\dfrac{1}{4m+\dfrac{2}{m}}$\\[2mm]
        因为$m\geq 2$,故$S_{\triangle MNF_{2}}$面积的最大值为$\dfrac{1}{9}$。
        \newpage
    \end{multicols}

\end{document}
