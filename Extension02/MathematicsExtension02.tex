% !TEX program = xelatex
\documentclass[UTF8]{ctexart}

\RequirePackage{inputenc}
\RequirePackage{fontspec}
\RequirePackage{xeCJK}

\RequirePackage{amsmath}
\RequirePackage{amssymb}
\RequirePackage{mathpazo}
\RequirePackage{pgfplots}
\RequirePackage{tikz}
\RequirePackage{tkz-euclide}

\usetkzobj{all}

\RequirePackage{subfigure}

\setmainfont{Times New Roman}

\setCJKmainfont{等线}
\setCJKsansfont{等线}
\setCJKmonofont{等线}

\let\nvec\vec
\def\vec#1{\nvec{\vphantom b\smash{#1}}}

\newcommand*{\dif}{\mathop{}\!\mathrm{d}}

\renewcommand{\Re}{\operatorname{Re}}
\renewcommand{\Im}{\operatorname{Im}}
\newcommand{\arccot}{\operatorname{arccot}}

\makeatletter
\newcommand{\rmnum}[1]{\romannumeral #1}
\newcommand{\Rmnum}[1]{\expandafter\@slowromancap\romannumeral #1@}
\makeatother

\usepackage{geometry}
\geometry
{
    left=1.25in,
    right=1.25in,
    top=1in,
    bottom=1in
}

\title{数学扩展研究\Rmnum{2}~~-~~四面体}
\author{李宇轩}
\date{2020.03.12}

\begin{document}

\maketitle

\newpage

\tableofcontents

\newpage

\setlength{\parindent}{0pt}

\section{四面体}

\subsection{四面体的符号约定}
    我们首先进行符号约定,若没有特殊说明,这些符号将在后文表达相同的含义。\\[3mm]
    我们依照下方表格的规定进行符号约定:\vspace{5pt}
    \begin{table}[h]
        \begin{center}
            \begin{tabular}{l|l}
                \hline
                符号~~~~~~~~&含义~~~~~~~~~~~~~~~~~~~~~~~~~~~~~~~~~~~~~~~~~~~~~~~~~~~~~~~~~~~~~~~~~~~~~~~~~~~~\\ \hline
                $\alpha$&线线角(直线$OB$和直线$OC$所成角)\\ \hline
                $\beta$&线线角(直线$OC$和直线$OA$所成角)\\ \hline
                $\gamma$&线线角(直线$OA$和直线$OB$所成角)\\ \hline
                $\theta_{a}$&线面角(直线$OA$和平面$ABC$所成角)\\ \hline
                $\theta_{b}$&线面角(直线$OB$和平面$ABC$所成角)\\ \hline
                $\theta_{c}$&线面角(直线$OC$和平面$ABC$所成角)\\ \hline
                $A$&面面角(平面$OAC$和平面$OAB$所成角)\\ \hline
                $B$&面面角(平面$OBA$和平面$OBC$所成角)\\ \hline
                $C$&面面角(平面$OCB$和平面$OCA$所成角)\\ \hline
            \end{tabular}
            \caption{四面体的符号约定}
        \end{center}
    \end{table}\\
    我们将下方图片所示的四面体作为参考:
    \begin{figure}[h]
        \begin{center}
            \begin{tikzpicture}
                \tkzInit[xmin=-0.5,xmax=8.5,ymin=-1.5,ymax=4.5]
                \tkzClip
                \tkzDefPoint(5,-1){A}
                \tkzDefPoint(8,0.5){B}
                \tkzDefPoint(0,0){C}
                \tkzDefPoint(4,4){O}

                \tkzLabelPoints[below](A)
                \tkzLabelPoints[right](B)
                \tkzLabelPoints[left](C)
                \tkzLabelPoints[above](O)

                \tkzDrawSegments(A,B)
                \tkzDrawSegments[dashed](B,C)
                \tkzDrawSegments(C,A)
                \tkzDrawSegments(O,A)
                \tkzDrawSegments(O,B)
                \tkzDrawSegments(O,C)
                
            \end{tikzpicture}
            \caption{四面体的示意图}
        \end{center}
    \end{figure}\\

\newpage

\subsection{四面体的空间角公式}
    本章将研究四面体中,线线角,线面角,面面角,三者间的数量关系。

\subsubsection{四面体空间角基本公式}
    四面体空间角基本公式(线线角形式):
    \begin{large}
        \begin{align*}
            \cos{\alpha}&=\cos{\beta}\cdot\cos{\gamma}+\sin{\beta}\cdot\sin{\gamma}\cdot\cos{A}\\[3mm]
            \cos{\beta}&=\cos{\gamma}\cdot\cos{\alpha}+\sin{\gamma}\cdot\sin{\alpha}\cdot\cos{B}\\[3mm]
            \cos{\gamma}&=\cos{\alpha}\cdot\cos{\beta}+\sin{\alpha}\cdot\sin{\beta}\cdot\cos{C}
        \end{align*}
    \end{large}\\
    四面体空间角基本公式(面面角形式):
    \begin{large}
        \begin{align*}
            \cos{A}&=\frac{cos{\alpha}-\cos{\beta}\cdot\cos{\gamma}}{\sin{\beta}\cdot\sin{\gamma}}\\[3mm]
            \cos{B}&=\frac{cos{\beta}-\cos{\gamma}\cdot\cos{\alpha}}{\sin{\gamma}\cdot\sin{\alpha}}\\[3mm]
            \cos{C}&=\frac{cos{\gamma}-\cos{\alpha}\cdot\cos{\beta}}{\sin{\alpha}\cdot\sin{\beta}}
        \end{align*}
    \end{large}\\
    在四面体$O-ABC$中,在$OA$上任取一点$D$。\\[2mm]
    在四面体$O-ABC$中,取$OB$上一点$E$使得$ED\perp OA$。\\[2mm]
    在四面体$O-ABC$中,取$OC$上一点$F$使得$FD\perp OA$。\\[4mm]
    四面体$O-ABC$及其辅助线如下图所示:\vspace{-5pt}
    \begin{figure}[h!]
        \begin{center}
            \begin{tikzpicture}
                \tkzInit[xmin=-0.5,xmax=6.9,ymin=-1.5,ymax=3.7]
                \tkzClip
                \tkzDefPoint(4,-0.8){A}
                \tkzDefPoint(6.4,0.4){B}
                \tkzDefPoint(0,0){C}
                \tkzDefPoint(3.2,3.2){O}

                \tkzLabelPoints[below](A)
                \tkzLabelPoints[right](B)
                \tkzLabelPoints[left](C)
                \tkzLabelPoints[above](O)

                \tkzDrawSegments(A,B)
                \tkzDrawSegments[dashed](B,C)
                \tkzDrawSegments(C,A)
                \tkzDrawSegments(O,A)
                \tkzDrawSegments(O,B)
                \tkzDrawSegments(O,C)

                \tkzDefBarycentricPoint(O=7,A=3)
                \tkzGetPoint{D}

                \tkzDefBarycentricPoint(O=4,B=6)
                \tkzGetPoint{E}

                \tkzDefBarycentricPoint(O=3,C=7)
                \tkzGetPoint{F}

                \tkzDrawSegments(D,E)
                \tkzDrawSegments(D,F)
                \tkzDrawSegments[dashed](E,F)

                \tkzLabelPoints[above right](D)
                \tkzLabelPoints[above right](E)
                \tkzLabelPoints[above left](F)

                \tkzDefBarycentricPoint(O=9,A=1)
                \tkzGetPoint{M}

                \tkzDefBarycentricPoint(O=15,C=1)
                \tkzGetPoint{N}

                \tkzDrawArc[color=black](O,M)(E)
                \tkzDrawArc[color=black](O,N)(D)

                \tkzDefShiftPoint[O](-150:0.45){b}
                \tkzLabelPoint(b){$\beta$}

                \tkzDefShiftPoint[O](-80:0.3){g}
                \tkzLabelPoint(g){$\gamma$}
            \end{tikzpicture}
            \caption{四面体空间角基本公式的示意图}
        \end{center}
    \end{figure}

\newpage

    在$\triangle DEF$中根据余弦定理可得:
    \begin{align}
        EF^2=DE^2+DF^2-2\cdot DE\cdot DF\cdot\cos{A}
    \end{align}\\
    在$\triangle OEF$中根据余弦定理可得:
    \begin{align}
        EF^2=OE^2+OF^2-2\cdot OE\cdot OF\cdot\cos{\alpha}
    \end{align}\\
    在$\triangle ODE$中角$ODE$是直角,根据勾股定理可得:
    \begin{align}
        OE^2=OD^2+DE^2
    \end{align}\\
    在$\triangle ODE$中角$ODF$是直角,根据勾股定理可得:
    \begin{align}
        OF^2=OD^2+DF^2
    \end{align}\\
    将式(3)式(4)代入式(2)中,消去$OE^2$和$OF^2$:
    \begin{align}
        &EF^2=DE^2+DF^2-2\cdot OE\cdot OF\cdot\cos{\alpha}\\[3mm]
        &EF^2=(OD^2+DE^2)+(OD^2+DF^2)-2\cdot OE\cdot OF\cdot\cos{\alpha}\\[3mm]
        &EF^2=2OD^2+DE^2+DF^2-2\cdot OE\cdot OF\cdot\cos{\alpha}
    \end{align}\\
    将式(7)和式(1)相减,整理可得:\vspace{3pt}
    \begin{align}
        &\left(2OD^2+DE^2+DF^2-2\cdot OE\cdot OF\cdot\cos{\alpha}\right)-\left(DE^2+DF^2-2\cdot DE\cdot DF\cdot\cos{A}\right)=0\\[3mm]
        &\left(2OD^2+DE^2+DF^2-DE^2-DF^2\right)-\left(2\cdot OE\cdot OF\cdot\cos{\alpha}-2\cdot DE\cdot DF\cdot\cos{A}\right)=0\\[3mm]
        &~2\cdot OD^2-2\cdot OE\cdot OF\cdot\cos{\alpha}+2\cdot DE\cdot DF\cdot\cos{A}=0\\[3mm]
        &~2\cdot OE\cdot OF\cdot\cos{\alpha}=2OD^2+2\cdot DE\cdot DF\cdot\cos{A}\\[3mm]
        &~OE\cdot OF\cdot\cos{\alpha}=OD^2+DE\cdot DF\cdot\cos{A}
    \end{align}\\
    接下来将通过变形得到$\cos{\alpha}$和$\cos{A}$两者间的关系。

\newpage

    通过变形可以得到:
    \begin{align}
        &\cos{\alpha}=\frac{OD^2}{OE\cdot OF}+\frac{DE\cdot DF}{OE\cdot OF}\cdot\cos{A}\\[3mm]
        &\cos{\alpha}=\frac{OD}{OE}\cdot\frac{OD}{OD}+\frac{DE}{OE}\cdot\frac{DF}{OF}\cdot\cos{A}
    \end{align}\\
    根据直角三角形$\triangle ODF$可以得到:
    \begin{align}
        &\cos{\beta}=\frac{OD}{OF}~~~~~~~~\sin{\beta}=\frac{DF}{OF}
    \end{align}\\
    根据直角三角形$\triangle ODE$可以得到:
    \begin{align}
        &\cos{\gamma}=\frac{OD}{OE}~~~~~~~~\sin{\gamma}=\frac{DE}{OE}
    \end{align}\\
    将四组三角比代入可得:
    \begin{align}
        \cos{\alpha}=\cos{\gamma}\cdot\cos{\beta}+\sin{\gamma}\cdot\sin{\beta}\cdot\cos{A}
    \end{align}

\newpage

\subsubsection{四面体空间角导出公式}
    四面体空间角导出公式:
    \begin{large}
        \begin{align*}
            &\sin{A}\cdot\sin{\beta}\cdot\sin{\gamma}=k\\[3mm]
            &\sin{B}\cdot\sin{\gamma}\cdot\sin{\alpha}=k\\[3mm]
            &\sin{C}\cdot\sin{\alpha}\cdot\sin{\beta}=k
        \end{align*}
    \end{large}\\
    其中代换变量$k$的取值为:\vspace{5pt}
    \begin{large}
        \begin{equation*}
            k=\sqrt{1-\cos^2{\alpha}-\cos^2{\beta}-\cos^2{\gamma}+2\cdot\cos{\alpha}\cdot\cos{\beta}\cdot\cos{\gamma}}
        \end{equation*}
    \end{large}\\[3mm]
    代入四面体空间角基本公式可得:\vspace{5pt}
    \setcounter{equation}{0}
    \begin{align}
        \sin{A}
        &=\sqrt{1-\cos{A}^2}\\[3mm]
        &=\sqrt{1-\left(\frac{\cos{\alpha}-\cos{\beta}\cdot\cos{\gamma}}{\sin{\beta}\cdot\sin{\gamma}}\right)^2}\\[3mm]
        &=\sqrt{1-\frac{\left(\cos{\alpha}-\cos{\beta}\cdot\cos{\gamma}\right)^2}{\sin^2{\beta}\cdot\sin^2{\gamma}}}\\[3mm]
        &=\sqrt{1-\frac{\cos^2{\alpha}+\cos^2{\beta}\cdot\cos^2{\gamma}-2\cdot\cos{\alpha}\cdot\cos{\beta}\cdot\cos{\gamma}}{\sin^2{\beta}\cdot\sin^2{\gamma}}}\\[3mm]
        &=\sqrt{\frac{\sin^2{\beta}\cdot\sin^2{\gamma}-\cos^2{\beta}\cdot\cos^2{\gamma}-\cos^2{\alpha}+2\cdot\cos{\alpha}\cdot\cos{\beta}\cdot\cos{\gamma}}{\sin^2{\beta}\cdot\sin^2{\gamma}}}
    \end{align}\\
    进一步代换可以得到:\vspace{5pt}
    \begin{align}
        \sin{A}
        &=\frac{\sqrt{\left(1-\cos^2{\beta}\right)\cdot\left(1-\cos^2{\gamma}\right)-\cos^2{\beta}\cdot\cos^2{\gamma}-\cos^2{\alpha}+2\cdot\cos{\alpha}\cdot\cos{\beta}\cdot\cos{\gamma}}}{\sin{\beta}\cdot\sin{\gamma}}\\[3mm]
        &=\frac{\sqrt{1-\cos^2{\beta}-\cos^2{\gamma}-\cos^2{\alpha}+2\cdot\cos{\alpha}\cdot\cos{\beta}\cdot\cos{\gamma}}}{\sin{\beta}\cdot\sin{\gamma}}\\[3mm]
        &=\frac{\sqrt{1-\cos^2{\alpha}-\cos^2{\beta}-\cos^2{\gamma}+2\cdot\cos{\alpha}\cdot\cos{\beta}\cdot\cos{\gamma}}}{\sin{\beta}\cdot\sin{\gamma}}
    \end{align}\\

\newpage

    定义代换变量$k$:
    \begin{align}
        k=\sqrt{1-\cos^2{\alpha}-\cos^2{\beta}-\cos^2{\gamma}+2\cdot\cos{\alpha}\cdot\cos{\beta}\cdot\cos{\gamma}}
    \end{align}\\
    代入代换变量$k$:
    \begin{align}
        &\sin{A}=\frac{k}{\sin{\beta}\cdot\sin{\gamma}}\\[3mm]
        &\sin{A}\cdot\sin{\beta}\cdot\sin{\gamma}=k
    \end{align}\vspace{5pt}

\subsubsection{四面体空间角比例公式}
    四面体空间角比例公式:
    \begin{large}
        \begin{equation*}
            \frac{\sin{A}}{\sin{\alpha}}=\frac{\sin{B}}{\sin{\beta}}=\frac{\sin{C}}{\sin{\gamma}}=\frac{k}{\sin{\alpha}\cdot\sin{\beta}\cdot\sin{\gamma}}
        \end{equation*}
    \end{large}\\
    其中代换变量$k$的取值为:\vspace{5pt}
    \begin{large}
        \begin{equation*}
            k=\sqrt{1-\cos^2{\alpha}-\cos^2{\beta}-\cos^2{\gamma}+2\cdot\cos{\alpha}\cdot\cos{\beta}\cdot\cos{\gamma}}
        \end{equation*}
    \end{large}\\
    在四面体空间角导出公式两边同除可得:\vspace{5pt}
    \setcounter{equation}{0}
    \begin{align}
        &\sin{A}\cdot\sin{\beta}\cdot\sin{\gamma}=k\\[3mm]
        &\frac{\sin{A}}{\sin{\alpha}}=\frac{k}{\sin{\alpha}\cdot\sin{\beta}\cdot\sin{\gamma}}\\[8mm]
        &\sin{B}\cdot\sin{\gamma}\cdot\sin{\alpha}=k\\[3mm]
        &\frac{\sin{B}}{\sin{\beta}}=\frac{k}{\sin{\alpha}\cdot\sin{\beta}\cdot\sin{\gamma}}\\[8mm]
        &\sin{C}\cdot\sin{\alpha}\cdot\sin{\beta}=k\\[3mm]
        &\frac{\sin{C}}{\sin{\gamma}}=\frac{k}{\sin{\alpha}\cdot\sin{\beta}\cdot\sin{\gamma}}
    \end{align}
    

    


\end{document}
