% !TEX program = xelatex
\documentclass[UTF8]{ctexart}

\RequirePackage{inputenc}
\RequirePackage{fontspec}
\RequirePackage{xeCJK}

\RequirePackage{amsmath}
\RequirePackage{amssymb}
\RequirePackage{mathpazo}
\RequirePackage{pgfplots}
\RequirePackage{tikz}
\RequirePackage{tkz-euclide}

\RequirePackage[hidelinks]{hyperref}

\RequirePackage{multicol}

\usetikzlibrary{calc}
\usetikzlibrary{intersections}
\usetikzlibrary{angles}
\usetkzobj{all}

\RequirePackage{subfigure}

\setmainfont{Times New Roman}

\setCJKmainfont{等线}
\setCJKsansfont{等线}
\setCJKmonofont{等线}

\let\nvec\vec
\def\vec#1{\nvec{\vphantom b\smash{#1}}}

\renewcommand\parallel{{\mathrel{/\mskip-2.5mu/}}}

\newcommand*{\dif}{\mathop{}\!\mathrm{d}}

\renewcommand{\Re}{\operatorname{Re}}
\renewcommand{\Im}{\operatorname{Im}}
\newcommand{\arccot}{\operatorname{arccot}}

\newcommand{\rnum}[1]{\uppercase\expandafter{\romannumeral #1\relax}}

\usepackage{geometry}
\geometry
{
    left=1.25in,
    right=1.25in,
    top=1in,
    bottom=1in
}

\title{关于第09期题目的讨论}
\author{数学学习小组}
\date{2020.08.06}

\begin{document}

\maketitle

\newpage

\tableofcontents

\newpage

\setlength{\parindent}{0pt}
\setlength{\columnseprule}{0.4pt}
\setlength{\columnsep}{40pt}

\section{题目09-1}
    本题来源于第09期(2020.08.06)小组讨论题中,原题号为2019年崇明一模20题的第3小问。\\[3mm]
    对于椭圆$\Gamma$:
    \begin{large}
        \begin{equation*}
            \frac{x^2}{16}+\frac{y^2}{4}=1
        \end{equation*}
    \end{large}\\
    设椭圆$\Gamma$的短轴的上端点为$B_1$。\\[3mm]
    设椭圆$\Gamma$的短轴的下端点为$B_2$。\\[3mm]
    设点$P$为椭圆$\Gamma$上异于$B_1,B_2$的动点。\\[3mm]
    设平面上一点$R$满足$RB_1\perp PB_1$且$RB_2\perp PB_2$。
    \begin{figure}[h]
        \begin{center}
            \qquad\qquad\qquad
            \begin{tikzpicture}[scale=0.8]
                \tkzDefPoint(0,0){O}

                \tkzDefPoint(-8,0){X1}
                \tkzDefPoint(+8,0){X2}

                \tkzDefPoint(0,-6){Y1}
                \tkzDefPoint(0,+6){Y2}

                \tkzDrawSegments[->](X1,X2)
                \tkzDrawSegments[->](Y1,Y2)

                \tkzLabelPoint[above](X2) {$x$}
                \tkzLabelPoint[right](Y2) {$y$}
                
                \draw[thick](0,0) ellipse(4 and 2);

                \tkzDefPoint(0,+2){B1}
                \tkzDefPoint(0,-2){B2}

                \tkzDefPoint(3,1.323){P}
                \tkzDefPoint(-0.805,-1.445){R}

                \tkzDrawSegment[add=0.5 and 0.7](P,R)

                \tkzDrawSegments(P,B1 R,B1 P,B2 R,B2)

                \tkzMarkRightAngle[size=0.2](P,B1,R)
                \tkzMarkRightAngle[size=0.2](P,B2,R)

                \tkzDrawPoints[fill=white](P,B1,B2,R)

                \tkzLabelPoint[above right](B1){$B_1$}
                \tkzLabelPoint[below right](B2){$B_2$}

                \tkzLabelPoint[above](P){$P$}
                \tkzLabelPoint[left=2pt](R){$R$}

            \end{tikzpicture}
            \caption{题目09-01的示意图}
        \end{center}
    \end{figure}\\
    证明$\triangle PB_1B_2$和$\triangle RB_1B_2$的面积比为定值。

\newpage

\subsection{第一种解法}
    \begin{center}
        整理者:施安然
    \end{center}
    \begin{multicols}{2}
        \small
        \textbf{核心思路:联立直线求横坐标的关系。}\\[5mm]
        设点$P(x_1, y_1)$,设$R(x_2, y_2)$。\\[5mm]
        因为$B_1(+2,0)$:$k_{PB_1} = \dfrac{y_1-2}{x_1}$\\[5mm]
        因为$B_2(-2,0)$:$k_{PB_2} = \dfrac{y_1+2}{x_2}$\\[5mm]
        由于$PB_1\perp RB_1$:$k_{RB_1} = -\dfrac{x_1}{y_1-2}$\\[5mm]
        由于$PB_2\perp RB_2$:$k_{RB_2} = -\dfrac{x_1}{y_1+2}$\\[5mm]
        故直线$RB_1: y = -\dfrac{x_1}{y_1-2}\cdot x + 2$\\[5mm]
        故直线$RB_2: y = -\dfrac{x_1}{y_1+2}\cdot x - 2$\\[5mm]
        由于$R$在直线上$RB_1: y = -\dfrac{x_1}{y_1-2}\cdot x_2 + 2$\\[5mm]
        由于$R$在直线上$RB_2: y = -\dfrac{x_1}{y_1+2}\cdot x_2 - 2$\\[8mm]
        联立两式可以得到: \\[5mm]
        $\dfrac{x_1}{y_1-2}\cdot x_2 - 2 = \dfrac{x_1}{y_1+2}\cdot x_2 + 2 $\\[100mm]
        $x_2=\dfrac{4}{\dfrac{x_1}{y_1-2}-\dfrac{x_1}{y_1+2}}$\\[5mm]
        $x_2=\dfrac{4}{\dfrac{4x_1}{y_1^2-4}}$\\[5mm]
        $x_2=\dfrac{4}{\dfrac{4x_1}{y_1^2-4}}$\\[5mm]
        $x_2=\dfrac{{y_1}^2 - 4}{x_1}$\\[8mm]
        由于两个三角形同底:\\[5mm]
        $\dfrac{S_\triangle PB_1B_2}{S_\triangle RB_1B_2} = \dfrac{|x_1|}{|x_2|}$\\[5mm]
        $\dfrac{S_\triangle PB_1B_2}{S_\triangle RB_1B_2} = \left|\dfrac{x_1^2}{y_1^2-4}\right|$\\[5mm]
        $\dfrac{S_\triangle PB_1B_2}{S_\triangle RB_1B_2} = 4$
        \newpage
    \end{multicols}

\newpage

\subsection{第二种解法}
    \begin{center}
        整理者:乔君毅
    \end{center}
    \begin{multicols}{2}
        \small
        \textbf{核心思路:四点共圆。}\\[5mm]
        因为$RB_{1} \perp PB_{1}$,同时$RB_{2}\perp PB_{2}$,故四点共圆。\\[5mm]
        设点$P(x_{1},y_{1})$,设点$R(X_{2},Y_{2})$。\\[5mm]
        设圆的圆心为$(a,b)$,设圆的半径为$r$。\\[5mm]
        设圆的方程为:$(x-a)^{2}+(y-b)^{2}= r^{2}$ 。\\[5mm]
        因为$B_{1}(0,2),B_{2}(0,-2)$在圆上:\\[5mm]
        \begin{math}
            \begin{cases}
                ~a^{2}+(b-2)^{2}= r^{2}\\[1mm]
                ~a^{2}+(b+2)^{2}= r^{2}\\[1mm]
            \end{cases}
        \end{math}\\[5mm]
        两式相减得$b=0$,所以$a^{2}+4= r^{2}$,故圆心$(a,0)$。\\[5mm]
        将点$P$带入圆中得$(x_{1}-a)^{2}+y_{1}^{2}=r^{2}$。\\[5mm]
        由于$P$在椭圆上$y_1^2=4-\dfrac{x_1^2}{4}$。\\[5mm]
        因此原式可以变为:\\[5mm]
        $(x_{1}-a)^{2}+4-\dfrac{x_{1}^{2} }{4}=a^{2}+4$\\[5mm]
        ${x_{1}}^{2} -2ax_{1}+a^{2} +4-\dfrac{x_{1}^{2} }{4}=a^{2}+4$ \\[5mm]
        ${x_{1}}^{2} -2ax_{1}-\dfrac{x_{1}^{2} }{4}=0$ \\[5mm]
        $\dfrac{3x_1^2}{4}-2ax_1=0$ \\[5mm]
        $x_1=\dfrac{3a}{8}$~~~~或~~~~$x_1=0$(舍)\\[100mm]
        因为$RP$为直径,其中点为圆心。\\[5mm]
        所以$x_{1}+x_{2}=2a$,故$x_2=-\dfrac{2}{3}a$。\\[5mm]
        所以$\dfrac{S_{\triangle}PB_{1}B_{2} }{S_{\triangle}RB_{1}B_{2} } =\dfrac{\left | x_{1} \right | }{\left | x_{2} \right | }=4 $
        \newpage
    \end{multicols}

\end{document}
