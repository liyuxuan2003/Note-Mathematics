% !TEX program = xelatex
\documentclass[UTF8]{ctexart}

\RequirePackage{inputenc}
\RequirePackage{fontspec}
\RequirePackage{xeCJK}

\RequirePackage{amsmath}
\RequirePackage{amssymb}
\RequirePackage{mathpazo}
\RequirePackage{pgfplots}
\RequirePackage{tikz}
\RequirePackage{tkz-euclide}

\RequirePackage[hidelinks]{hyperref}

\RequirePackage{multicol}

\usetikzlibrary{calc}
\usetikzlibrary{intersections}
\usetikzlibrary{angles}
\usetkzobj{all}

\RequirePackage{subfigure}

\setmainfont{Times New Roman}

\setCJKmainfont{等线}
\setCJKsansfont{等线}
\setCJKmonofont{等线}

\let\nvec\vec
\def\vec#1{\nvec{\vphantom b\smash{#1}}}

\renewcommand\parallel{{\mathrel{/\mskip-2.5mu/}}}

\newcommand*{\dif}{\mathop{}\!\mathrm{d}}

\renewcommand{\Re}{\operatorname{Re}}
\renewcommand{\Im}{\operatorname{Im}}
\newcommand{\arccot}{\operatorname{arccot}}

\newcommand{\rnum}[1]{\uppercase\expandafter{\romannumeral #1\relax}}

\usepackage{geometry}
\geometry
{
    left=1.25in,
    right=1.25in,
    top=1in,
    bottom=1in
}

\title{关于第08期题目的讨论}
\author{数学学习小组}
\date{2020.07.25}

\begin{document}

\maketitle

\newpage

\tableofcontents

\newpage

\setlength{\parindent}{0pt}
\setlength{\columnseprule}{0.4pt}
\setlength{\columnsep}{40pt}

\section{题目08-1}
    本题来源于第08期(2020.07.25)小组讨论题中,原题号为2018年宝山一模20题的第3小问。\\[3mm]
    对于椭圆$\Gamma$:
    \begin{large}
        \begin{equation*}
            \frac{x^2}{4}+\frac{y^2}{3}=1
        \end{equation*}
    \end{large}\\
    设椭圆$\Gamma$的右焦点为$F$,过右焦点$F$的直线$l$交$\Gamma$于$A,B$两点。\\[3mm]
    设点$A$在直线$x=4$上的射影点为$D$。\\[3mm]
    设点$B$在直线$x=4$上的射影点为$E$。
    \begin{figure}[h]
        \begin{center}
            \qquad\qquad\qquad
            \begin{tikzpicture}[scale=0.8]
                \tkzDefPoint(0,0){O}

                \tkzDefPoint(-8,0){X1}
                \tkzDefPoint(+8,0){X2}

                \tkzDefPoint(0,-6){Y1}
                \tkzDefPoint(0,+6){Y2}

                \tkzDrawSegments[->](X1,X2)
                \tkzDrawSegments[->](Y1,Y2)

                \tkzLabelPoint[above](X2) {$x$}
                \tkzLabelPoint[right](Y2) {$y$}
                
                \draw[thick](0,0) ellipse(3.4641 and 3);

                \tkzDefPoint(6.9282,0){P}
                \tkzDefShiftPoint[P](0,-2.966){D}
                \tkzDefShiftPoint[P](0,+1.703){E}
                
                \tkzDefPoint(+0.517,-2.966){A}
                \tkzDefPoint(+2.852,+1.703){B}
                \tkzDefPoint(2,0){F}

                \tkzDrawSegment[add=0.4 and 0.4](A,B)
                \tkzDrawSegment[add=0.4 and 0.4](D,E)

                \tkzDrawSegment(B,E)
                \tkzDrawSegment(A,D)

                \tkzDrawPoints[fill=white](A,B,D,E,F)

                \tkzLabelPoint[below](A){$A$}
                \tkzLabelPoint[below=2pt](F){$F$}
                \tkzLabelPoint[below=2pt](B){$B$}
                \tkzLabelPoint[below left](O){$O$}

                \tkzLabelPoint[right](E){$E$}
                \tkzLabelPoint[right](D){$D$}
            \end{tikzpicture}
            \caption{题目08-01的示意图}
        \end{center}
    \end{figure}\\
    当直线$l$绕点$F$转动时,直线$AE$和直线$BD$是否相交于定点?

\newpage

\subsection{第一种解法}
    \begin{center}
        整理者:李宇轩
    \end{center}
    \begin{multicols}{2}
        \small
        \textbf{核心思路:特殊值法,利用向量平行。}\\[5mm]
        设$x=ty+1$,当$t$不存在时,舍。\\[4mm]
        取特殊值$t=0$,此时两直线相较于$S(2.5,0)$。\\[4mm]
        因此两条直线若交于定点,则必交于$S(2.5,0)$。\\[8mm]
        设点$A(x_1,y_1)$,故$D(4,y_1)$。\\[5mm]
        设点$B(x_2,y_2)$\hspace{1.5pt},故$E\hspace{1pt}(4,y_2)$。\\[5mm]
        联立直线和椭圆可得:\\[5mm]
        \begin{math}
            \begin{cases}
                ~3x^2+4y^2-12=0\\[1mm]
                ~x=ty+1\\[1mm]
            \end{cases}
        \end{math}\\[5mm]
        $(3t^2+4)\cdot x^2+6ty-9=0$\\[5mm]
        $y_1+y_2=\dfrac{-6t}{3t^2+4}$\\[5mm]
        $y_1\cdot y_2=\dfrac{-9}{3t^2+4}$\\[5mm]
        若$S$在直线$AE$上,则需满足$\overrightarrow{SA}\parallel\overrightarrow{SE}$。\\[5mm]
        若$S$在直线$BD$上,则需满足$\overrightarrow{SB}\parallel\overrightarrow{SD}$。\\[5mm]
        $\overrightarrow{SA}=(x_1-2.5~,~y_1)$\\[5mm]
        $\overrightarrow{SE}\hspace{1pt}=(1.5~,~y_2)$\\[100mm]
        $\hphantom{=}x_1y_2-2.5y_2-1.5y_1$\\[5mm]
        $=(ty_1+1)y_2-2.5y_2-1.5y_1$\\[5mm]
        $=ty_1y_2+y_1-2.5y_2-1.5y_1$\\[5mm]
        $=ty_1y_2-1.5(y_1+y_2)$\\[5mm]
        $=ty_1y_2-1.5(y_1+y_2)$\\[5mm]
        $=\dfrac{-9t^2}{3t^2+4}+\dfrac{9t^2}{3t^2+4}$\\[5mm]
        $=0$\\[8mm]
        因此$\overrightarrow{SA}\parallel\overrightarrow{SE}$成立。\\[5mm]
        同理$\overrightarrow{SB}\parallel\overrightarrow{SD}$成立。\\[5mm]
        因此两条直线相交于定点$S(2.5,0)$。
        \newpage
    \end{multicols}

\newpage

\section{题目08-2}
    本题来源于第08期(2020.07.25)小组讨论题中,原题号为2018年浦东一模20题的第3小问。\\[3mm]
    对于椭圆$\Gamma$:
    \begin{large}
        \begin{equation*}
            \frac{x^2}{4}+\frac{y^2}{1}=1
        \end{equation*}
    \end{large}\\
    已知一定点$E(2,1)$。\\[3mm]
    设点$A$为椭圆$\Gamma$的上端点。\\[3mm]
    设点$P$\hspace{1.0pt}为椭圆$\Gamma$上一动点。
    \begin{figure}[h]
        \begin{center}
            \qquad\qquad\qquad
            \begin{tikzpicture}[scale=0.8]
                \tkzDefPoint(0,0){O}

                \tkzDefPoint(-8,0){X1}
                \tkzDefPoint(+8,0){X2}

                \tkzDefPoint(0,-6){Y1}
                \tkzDefPoint(0,+6){Y2}

                \tkzDrawSegments[->](X1,X2)
                \tkzDrawSegments[->](Y1,Y2)

                \tkzLabelPoint[above](X2) {$x$}
                \tkzLabelPoint[right](Y2) {$y$}
                
                \draw[thick](0,0) ellipse(4 and 2);

                \tkzDefPoint(0,2){A}
                \tkzDefPoint(4,-2){E}
                \tkzDefPoint(-3.489,-0.978){P}
                \tkzDefPoint(4,0){Ex}
                \tkzDefPoint(0,-2){Ey}

                \tkzDrawPolygon(A,E,P)
                \tkzDrawSegments[dashed](E,Ex E,Ey)
                \tkzDrawPoints[fill=white](A,E,Ex,Ey,P)
                \tkzLabelPoint[above left](A){$A$}
                \tkzLabelPoint[below](E){$E$}
                \tkzLabelPoint[below left](P){$P$}
            \end{tikzpicture}
            \caption{题目08-02的示意图}
        \end{center}
    \end{figure}\\
    当$P$在椭圆上移动时,讨论$\triangle AEP$在不同的面积区间中存在的个数。

\newpage

\subsection{第一种解法}
    \begin{center}
        整理者:白昱轩
    \end{center}
    \begin{multicols}{2}
        \small
        \textbf{核心思路:标准方程,设平行直线求距离。}\\[5mm]
        已知$E(2,-1)$,同时$A(0,1)$,因此$k_{AE}=-1$。\\[5mm]
        故过$P$的平行于$AE$的直线可设为:\\[5mm]
        $l_P:y=-x+b$\\[5mm]
        联立直线和椭圆可得:\\[5mm]
        \begin{math}
            \begin{cases}
                ~y=-x+b\\[1mm]
                ~x^2+4y^2-4=0\\[1mm]
            \end{cases}
        \end{math}\\[5mm]
        $5x^2-8bx+4b^2-4=0$\\[5mm]
        $\Delta=-16b^2+80$\\[5mm]
        $\Delta=0 $\\[5mm]
        $ b = \pm \sqrt{5}$\\[5mm]
        当$b = +\sqrt{5}$时,$S_{\triangle AEP}= \sqrt{5}-1$\\[5mm]
        当$b = -\sqrt{5}$时,$S_{\triangle AEP}= \sqrt{5}+1$\\[5mm]
        存在$4$个:满足$S_{\triangle AEP} \in \left(0,\sqrt{5}-1\right)$\\[5mm]
        存在$3$个:满足$S_{\triangle AEP} = \sqrt{5}-1$\\[5mm]
        存在$2$个:满足$S_{\triangle AEP} \in \left(\sqrt{5}-1,\sqrt{5}+1\right)$\\[5mm]
        存在$1$个:满足$S_{\triangle AEP}= \sqrt{5}+1$\\[5mm]
        存在$0$个:满足$S_{\triangle AEP} \in \left(\sqrt{5}+1,\infty \right)$\\[50mm]
        \newpage
    \end{multicols}

\newpage

\subsection{第二种解法}
    \begin{center}
        整理者:白昱轩
    \end{center}
    \begin{multicols}{2}
        \small
        \textbf{核心思路:参数方程,计算点到直线距离。}\\[5mm]
        已知$E(2,-1)$,同时$A(0,1)$,因此$k_{AE}=-1$。\\[5mm]
        故直线$AE$可表达为:\\[5mm]
        $l_{AE} : y = -x + 1$。\\[5mm]
        $l_{AE} : x+y-1=0$。\\[5mm]
        由参数方程设$P(2\cos\theta,\sin\theta)$,其中$\theta\in[0,2\pi)$。\\[5mm]
        根据点到直线距离公式:\\[5mm]
        $d = \dfrac{|2\cos\theta+\sin\theta-1|}{\sqrt{2}} \in \left[0,\dfrac{\sqrt{5}+1}{\sqrt{2}}\right]$\\[5mm]
        $d_1 = \dfrac{\sqrt{5}-1}{\sqrt{2}}$\\[5mm]
        $d_2 = \dfrac{\sqrt{5}+1}{\sqrt{2}}$\\[5mm]
        此外$AE=2\sqrt{2}$。\\[5mm]
        存在$4$个:满足$S_{\triangle AEP} \in \left(0,\sqrt{5}-1\right)$\\[5mm]
        存在$3$个:满足$S_{\triangle AEP} = \sqrt{5}-1$\\[5mm]
        存在$2$个:满足$S_{\triangle AEP} \in \left(\sqrt{5}-1,\sqrt{5}+1\right)$\\[5mm]
        存在$1$个:满足$S_{\triangle AEP}= \sqrt{5}+1$\\[5mm]
        存在$0$个:满足$S_{\triangle AEP} \in \left(\sqrt{5}+1,\infty \right)$
        \newpage
    \end{multicols}

\section{题目08-3}
    本题来源于第08期(2020.07.25)小组讨论题中,原题号为2018年静安一模20题的第3小问。\\[3mm]
    对于曲线$\Gamma$:
    \begin{large}
        \begin{equation*}
            \left|x^2-y^2\right|=1
        \end{equation*}
    \end{large}\\
    设动点$P$是曲线$\Gamma$上一动点,设直线$l_1:y=x$,设直线$l_2:y=-x$。\\[3mm]
    过动点$P$作直线$l_1$的平行线,交曲线$\Gamma$于点$Q$。\\[3mm]
    过动点$P$作直线$l_2$的平行线,交曲线$\Gamma$于点$R$。\\[3mm]
    \begin{figure}[h]
        \begin{center}
            \qquad\qquad\qquad
            \begin{tikzpicture}[scale=0.8]
                \tkzDefPoint(0,0){O}

                \tkzDefPoint(-8,0){X1}
                \tkzDefPoint(+8,0){X2}

                \tkzDefPoint(0,-6){Y1}
                \tkzDefPoint(0,+6){Y2}

                \tkzDrawSegments[->](X1,X2)
                \tkzDrawSegments[->](Y1,Y2)

                \tkzLabelPoint[above](X2) {$x$}
                \tkzLabelPoint[right](Y2) {$y$}
                
                \draw[thick] plot[variable=\t,domain=-55:55] ({+3*sec(\t)},{+3*tan(\t)});
                \draw[thick] plot[variable=\t,domain=-55:55] ({-3*sec(\t)},{+3*tan(\t)});

                \draw[thick] plot[variable=\t,domain=-55:55] ({+3*tan(\t)},{+3*sec(\t)});
                \draw[thick] plot[variable=\t,domain=-55:55] ({+3*tan(\t)},{-3*sec(\t)});

                \tkzDefPoint(-5,-5){L1D}
                \tkzDefPoint(+5,+5){L1U}

                \tkzDefPoint(+5,-5){L2D}
                \tkzDefPoint(-5,+5){L2U}

                \tkzDrawSegments[dashed](L1D,L1U L2D,L2U)

                \tkzDefPoint(-1.6,3.4){P}
                \tkzDefPoint(-3.4,1.6){Q}
                \tkzDefPoint(3.4,-1.6){R}

                \tkzDrawSegment[add=0 and 0.7](P,Q)
                \tkzDrawSegment[add=0 and 0.3](P,R)

                \tkzLabelPoint[above right](L1U){$l_1$}
                \tkzLabelPoint[above left](L2U){$l_2$}
                \tkzLabelPoint[above](P){$P$}
                \tkzLabelPoint[right](R){$R$}
                \tkzLabelPoint[left=2pt](Q){$Q$}

                \tkzDrawPoints[fill=white](P,Q,R)

            \end{tikzpicture}
            \caption{题目08-03的示意图}
        \end{center}
    \end{figure}\\
    证明三角形$\triangle PQR$的面积为定值,并求出该面积。

\newpage

\subsection{第一种解法}
    \begin{center}
        整理者:施安然
    \end{center}
    \begin{multicols}{2}
        \small
        \textbf{核心思路:联立求点,代入双曲线方程。}\\[5mm]
        由于$\left|x^2-y^2\right|=1$具有对称性。\\[5mm]
        因此只需要讨论$P$在$y^2-x^2=1$上的情况即可。\\[5mm]
        因此点$P(x_0,y_0)$满足$y_0^2-x_0^2=1$。\\[8mm]
        联立直线$PQ$和左右开口的双曲线:\\[5mm]
        \begin{math}
            \begin{cases}
                ~y - y_0 = x - x_0\\[2mm]
                ~x^2 - y^2 = 1\\[1mm]
            \end{cases}
        \end{math}\\[5mm]
        $x^2-(x-x_0+y_0)^2=1$\\[5mm]
        $x^2-(x^2+x_0^2+y_0^2-2xx_0+2xy_0-2x_0y_0)=1$\\[5mm]
        $x_0^2+y_0^2-2xx_0+2xy_0-2x_0y_0=-1$\\[5mm]
        $(x_0-y_0)^2-2x\cdot(x_0-y_0)=-1$\\[5mm]
        $2x\cdot(x_0-y_0)=(x_0-y_0)^2+1$\\[5mm]
        $2x\cdot(x_0-y_0)=(x_0-y_0)^2-(x_0^2-y_0^2)$\\[5mm]
        $2x\cdot(x_0-y_0)=(x_0-y_0)^2-(x_0+y_0)\cdot(x_0-y_0)$\\[5mm]
        $2x=(x_0-y_0)-(x_0+y_0)$\\[5mm]
        $x=-y_0$\\[100mm]
        因此$Q(-y_0, -x_0)$\\[5mm]
        同理$R(+y_0, +x_0)$\\[5mm]
        \begin{math}
            \begin{vmatrix}
                +x_0&+y_0&1\\
                +y_0&+x_0&1\\
                -y_0&-x_0&1\\
            \end{vmatrix}=2x_0^2-2y_0^2=-2
        \end{math}\\[5mm]
        $S_{\triangle PQR}=\dfrac{1}{2}\cdot\left|-2\right|=1$\\[5mm]
        $\because y^2 - x^2 = 1 $与$ x^2 - y^2 = 1 $共渐近线$ y = \pm x$\\[5mm]
        $\therefore P、Q$关于$y = -x$ 对称\\[5mm]
        $\therefore Q(-y_0, -x_0)$\\[5mm]
        同理:$R(y_0, x_0)$\\[5mm]
    \end{multicols}

\newpage

\subsection{第二种解法}
    \begin{center}
        整理者:施安然
    \end{center}
    \begin{multicols}{2}
        \small
        \textbf{核心思路:利用关于直线的对称点。}\\[5mm]
        由于$x^2-y^2=1$和$y^2-x^2=1$互为共轭双曲线。\\[5mm]
        故其渐近线均为$y=+x$和$y=-x$,设$P(x_0,y_0)$。\\[5mm]
        $P$关于$y=-x$的对称点为$Q(-y_0,-x_0)$。\\[5mm]
        $P$关于$y=+x$的对称点为$R(+y_0,+x_0)$。\\[5mm]
        \begin{math}
            \begin{vmatrix}
                +x_0&+y_0&1\\
                +y_0&+x_0&1\\
                -y_0&-x_0&1\\
            \end{vmatrix}=2x_0^2-2y_0^2=-2
        \end{math}\\[5mm]
        $S_{\triangle PQR}=\dfrac{1}{2}\cdot\left|-2\right|=1$\\[5mm]
        \newpage
    \end{multicols}

\newpage

\end{document}
